%%%%%%%%%%%%%%%%%%%%%%%%%%%%%%%%%%%%%%%%%%%%%%%%%%%%%%%%%%%%%%%%%%%%%%%%%%
%%
%% Use the following commands to convert each page to PNG.
%% Resolution is 400 dpi in both x and y axes.
%%
%%     $ pdftoppm 00-introduction.pdf introduction -png -rx 400 -ry 400
%%
%%%%%%%%%%%%%%%%%%%%%%%%%%%%%%%%%%%%%%%%%%%%%%%%%%%%%%%%%%%%%%%%%%%%%%%%%
\documentclass[12pt,aspectratio=169]{beamer}
\usepackage{beamerthemeshadow}
\usefonttheme{serif}
\usepackage[LGR,OT1]{fontenc}
\usepackage{epigrafica}
\usepackage{amsmath,amssymb,amsfonts}
\usepackage{graphicx}
\linespread{1.05}   
%% \usebackgroundtemplate%
%% {%
%%     \includegraphics[width=\paperwidth,height=\paperheight]{colored-background.jpg}%
%% }

\title{Introduction}
\institute{Makabulos Memorial High School}
\author{Robert Pascual\\
Oliver Gabor}
\date{SY 2020--2021}

\begin{document}
\maketitle

\begin{frame}
\frametitle{Objectives of this Course}\large

\begin{itemize}
\setlength\itemsep{0.2in}
\item We will be introduced to some elementary ideas of Hardware and Software
\item We will learn the concepts of Software Engineering.
\item We will be introduced to the Java Programming Language.
\end{itemize}
\end{frame}

\begin{frame}
\frametitle{Hardware and Software}\large

\noindent We shall find the answers to the following questions.

\begin{itemize}
\setlength\itemsep{0.2in}
\item What are the different forms of computers?
\item What are the parts of hardware?
\item What are the parts of software?
\end{itemize}
\end{frame}

\begin{frame}
\frametitle{Software Engineering}\large

\noindent Under Software Engineering, we will

\begin{itemize}
\setlength\itemsep{0.2in}
\item Learn how to write functional specifications.
\item Learn a few designing techniques and diagrams.
\item Learn about Software Life Cycle and how it affects
our implementations.
\end{itemize}
\end{frame}

\begin{frame}
\frametitle{Java Language}\large

Most of this course will be about the
Java Programming Language. We will
be learning about the following.


\begin{itemize}
\setlength\itemsep{0.2in}
\item The Installation and Setup of the Java SDK.
\item Java's Edit-Compile-Debug Cycle.
\item How to write simple Java Programs for the Desktop.
\end{itemize}
\end{frame}

\begin{frame}
\frametitle{What Is Expected of You?}\large

You are expected to

\begin{itemize}
\setlength\itemsep{0.2in}
\item Read books, do independent research, and view videos.
Do not rely on the lectures alone!
\item Work on the activities and submit online or by hard copy.
\item Write Java Programs.

\end{itemize}
\end{frame}

\begin{frame}
\frametitle{Study Help}\large

\begin{itemize}
\setlength\itemsep{0.2in}
\item PDF Copies of Slides.
\item Facebook Group. Don't hesitate to ask questions.
\item Regular Zoom Conferences. Notification will be on
      the Makabulos Memorial High School Facebook Page.
\item Your lecturers are Oliver Gabor and Robert Pascual.
\item Email: \textbf{mmhsxstaff@gmail.com}
\end{itemize}
\end{frame}

\end{document}

\begin{frame}
\frametitle{}

\begin{itemize}
\item 
\item 
\item 
\end{itemize}
\end{frame}

\begin{frame}
\frametitle{Find Out}

\begin{itemize}
\item Look at the problem.
\item Have you seen a similar problem before?
\item What facts do you have?
\item What do you know that is not stated in the problem?
\item What conceptual tools do you need?
\end{itemize}
\end{frame}

\begin{frame}
\frametitle{Choose A Strategy}

\begin{itemize}
\item How have you solved similar problems in the past?
\item What strategies do you know?
\item If unsure try any strategy which seems it will work.
\end{itemize}
\end{frame}

\begin{frame}
\frametitle{Solve It}
\begin{itemize}
\item Use the strategy you selected and work the problem.
\item Be careful and follow all rules of mathematics.
\end{itemize}
\end{frame}

\begin{frame}
\frametitle{Look Back}

\begin{itemize}
\item Did you answer the question?
\item Does the answer seem reasonable?
\item What insights did you learn?
\item Can you derive the solution differently?
\end{itemize}
\end{frame}

\begin{frame}
\frametitle{The Strategies}

\begin{itemize}
\item Compute or simplify
\item Use a formula
\item Look for patterns
\item Make a model or diagram
\item Make a table chart or list.
\item Guess, check, and revise.
\item Consider a simpler case.
\item Eliminate
\end{itemize}
\end{frame}

\begin{frame}
\frametitle{Problem 1}

If the length of the edge of a cube is increased by 50\%,
what is the percent
increase in the volume of the cube? 
\vspace{0.2in}

Express your answer to the nearest whole
number.

\end{frame}

\begin{frame}
\frametitle{Find Out}

This problem involves the volume of a cube. We
therefore expect to use the formula for the volume of
a cube $V = lwh$ or, since the cube has
equal sides $V = l^3$.

We also need to use percentages. Recall
that n\% means n/100 and thus
50\% is 50/100 or 1/2.

\end{frame}

\begin{frame}
\frametitle{Choose A Strategy}

\begin{itemize}
\item Use a formula
\item Compute or Simplify
\end{itemize}

\end{frame}

\begin{frame}
\frametitle{Solve It}

Let the original side of the cube be of length $l$.
The volume of the cube is thus $l^3$.
\vspace{0.2in}

If we increase the length $l_0$ by 50\%
it will become $l + 0.5l = 1.5l$.
\vspace{0.2in}

The volume of this cube is $(1.5l)^3$.
\end{frame}

\begin{frame}
\frametitle{Solve It}

Let's simplify ...

\begin{align*}
(1.5l^3) &= 1.5^3l^3\\
         &= 3.375l^3\\
\end{align*}

The increase in volume is

\begin{align*}
3.375l^3 - l^3 = 2.375l^3
\end{align*}
\vspace{0.2in}

The volume is increased by 237.5\%.

\end{frame}

\begin{frame}
\frametitle{Look Back}

If the side of a cube is increased by $p$\%
the volume will increase by $(1+p/100)^3 - 1$

\end{frame}

\begin{frame}
\frametitle{Problem 2}

What is the least natural number greater than 7 that has a remainder of 7 when
divided by 24 and also has a remainder of 7 when divided by 32?

\end{frame}

\begin{frame}
\frametitle{Find Out}

This problem involves remainders. We get remainders
when we do integer division.
\vspace{0.2in}

We note:
\vspace{0.2in}

if 15/7 = 2 r 1
then 15 = 7(2) + 1
\vspace{0.2in}

This can be done with any number.

\end{frame}

\begin{frame}
\frametitle{Find Out}

Therefore from what we know of remainders and
division we note that any given integer $a$ and
any divisor $q$ then 
$a = pq + r$ where $r < q$.
\end{frame}

\begin{frame}
\frametitle{Choose A Strategy}

This problem can be solved by

\begin{itemize}
\item Using an algebraic model.
\item Using a formula.
\end{itemize}

\end{frame}

\begin{frame}
\frametitle{Solve It}

Let $x = $ the unknown number. Then, from the given
data
\vspace{0.2in}

\begin{align*}
x &= 24s + 7\\
x &= 32t + 7\\
\end{align*}

Three unknowns but only two equations!
There is more than one solution possible.
\end{frame}

\begin{frame}
\frametitle{Solve It}

\begin{align*}
24s + 7 &= 32t + 7\\
24s     &= 32t\\
 3s     &= 4t\\
  s     &= \frac{4}{3}t
\end{align*}

We allow $t$ to be the independent parameter.
\end{frame}

\begin{frame}
\frametitle{Solve It}

\centerline{Table of $s$, $t$, and $x$.}

\begin{center}
\begin{tabular}[t]{ccc}
$t$  & $s$ & $x = 24s + 7$\\
\hline
3    & 4   & 103\\
6    & 8   & 199\\
9    & 12  & 295\\ 
\end{tabular}
\end{center}
\vspace{0.2in}

The smallest number which satisties the condition is
103.
\end{frame}

\begin{frame}
\frametitle{Solve it}

We check, just to make sure:
\vspace{0.2in}

\begin{align*}
103 &= 4(24) + 7\;\;\text{and}\\
103 &= 3(32) + 7\\
\end{align*}

\end{frame}

\begin{frame}
\frametitle{Look Back}

Can we derive the solution differently?
\vspace{0.2in}

Can we use Guess, Check, and Revise?
\vspace{0.2in}

Since we already have the answer, we do know
it should take only a few guesses to
find that solution.
\vspace{0.2in}

Let us try.
\end{frame}

\begin{frame}
\frametitle{Look Back}

Here we add 7 to multiples of 24
\vspace{0.2in}

\begin{center}
\begin{tabular}[t]{cc}
mult(24)    & add 7\\
\hline
24    & 31\\
48    & 55\\
72    & 79\\
96    & 103\\
120   & 127\\
144   & 151\\
168   & 175\\
\hline
\end{tabular}
\end{center}

\end{frame}

\begin{frame}
\frametitle{Look Back}

Here we add 7 to multiples of 32.
\vspace{0.2in}

\begin{center}
\begin{tabular}[t]{cc}
mult(32)    & add 7\\
\hline
32    & 39\\
64    & 71\\
96    & 103\\
128   & 135\\
160   & 167\\
192   & 199\\
224   & 231\\
\hline
\end{tabular}
\end{center}
\vspace{0.2 in}

The integer 103
is in both tables. Hence, one answer is 103.

\end{frame}

\begin{frame}
\frametitle{Problem}
The ratio of the number of cans of cola soda to 
lemon-lime soda to cherry soda
consumed at a graduation party was 12:3:10. 
If a total of 150 cans of these three
flavors of soda were consumed, how many cans were lemon-lime soda?
\end{frame}

\begin{frame}
\frametitle{Find Out}

This is a problem in ratios.
\vspace{0.2 in}

Recall what we studied in ratios.
\vspace{0.2 in}

Try thinking of simple ratio problems to
familiarize ourselves with the procedure.

\end{frame}

\begin{frame}
\frametitle{Find Out}

If there are 2 cans of cola for every 3 cans of lemon-line
then the ratio is 2:3.
\vspace{0.2in}

Note $2 + 3 = 5$.

\end{frame}

\begin{frame}
\frametitle{Find Out}

If there are 2 cans of cola for every 3 cans of lemon-lime and
4 cans of cherry, then the ratio is 2:3:4.

If there are 4 cans of cola then there are

\begin{itemize}
\item 6 cans lemon-lime
\item 8 cans cherry.
\end{itemize}

Total cans is 18.
\vspace{0.2 in}

Note: $2 + 3 + 4 = 9$ and $18/9 = 2$.

\end{frame}

\begin{frame}
\frametitle{Choose A Strategy}

\begin{itemize}
\item Consider a simpler case. (During Find Out)
\item Compute or simplify.
\end{itemize}

\end{frame}

\begin{frame}
\frametitle{Solve It}

$$12 + 3 + 10 = 25$$

$$150/25 = 6$$

\end{frame}

\begin{frame}
\frametitle{Solve It}

Therefore there are:

\begin{itemize}
\item 72 cola sodas
\item 18 lemon-lime sodas
\item 60 cherry sodas
\end{itemize}
\vspace{0.2in}

Check: $72 + 18 + 60 = 150$

\end{frame}

\begin{frame}
\frametitle{Look Back}

Have we answered the question?
\vspace{0.2 in}

The answer to the question is 18.

\end{frame}

\begin{frame}
\frametitle{Look Back}

Can it be solved using algebra?

Can we generalize this?


\end{frame}

\end{document}


\begin{frame}
\frametitle{}
\end{frame}

\begin{frame}
\frametitle{}
\end{frame}

\begin{frame}
\frametitle{}
\end{frame}

\begin{frame}
\frametitle{}
\end{frame}

\begin{frame}
\frametitle{}
\end{frame}

\section{The Format}

Each stated and solved problem should have the following parts.

\begin{enumerate}
\item \textbf{Problem:} This is the statement
of the problem.
\item \textbf{Solution:} The solution. It is composed
of the four steps we follow when solving mathematical problems.

\begin{itemize}
\item \textbf{Find Out}
\item \textbf{Select a Strategy}
\item \textbf{Solve It}
\item \textbf{Look Back}
\end{itemize}

\end{enumerate}

\section{Problem}

\subsection{Solution}
\subsubsection{Find Out}
\subsubsection{Choose A Strategy}

\subsubsection{Solve It}

\subsubsection{Look Back}

\section{Problem}

\subsection{Solution}

\subsubsection{Find Out}
\section{Problem}

\subsection{Solution}
\subsubsection{Find Out}
\subsubsection{Choose A Strategy}
\subsubsection{Solve It}
\subsubsection{Look Back}

\section{Problem}

\subsection{Solution}
\subsubsection{Find Out}
\subsubsection{Choose A Strategy}
\subsubsection{Solve It}
\subsubsection{Look Back}

\section{Problem}

\subsection{Solution}
\subsubsection{Find Out}
\subsubsection{Choose A Strategy}
\subsubsection{Solve It}
\subsubsection{Look Back}

\end{document}

\section{Problem}

\subsection{Solution}
\subsubsection{Find Out}
\subsubsection{Choose A Strategy}
\subsubsection{Solve It}
\subsubsection{Look Back}

\section{Problem}

\subsection{Solution}
\subsubsection{Find Out}
\subsubsection{Choose A Strategy}
\subsubsection{Solve It}
\subsubsection{Look Back}

\section{Problem}

\subsection{Solution}
\subsubsection{Find Out}
\subsubsection{Choose A Strategy}
\subsubsection{Solve It}
\subsubsection{Look Back}




